\documentclass[11pt,spanish]{article} % Tipo y tamaño de letra del documento.


\usepackage[utf8]{inputenc}
\usepackage{subfiles}
\usepackage{biblatex}
\addbibresource{references.bib}
\usepackage{multicol}
\usepackage{amsfonts}
\usepackage{blindtext}
\usepackage{mathrsfs}
\usepackage{amsmath}
\usepackage{siunitx}
\usepackage{centernot}
\usepackage[shortlabels]{enumitem}
\usepackage{subfig}
\usepackage{datetime}
\usepackage{listingsutf8}
\usepackage[spanish]{babel}
\usepackage{tikz}
\usepackage{hyperref}
\usepackage[vlined,ruled,linesnumbered]{algorithm2e}
\usepackage{listings}
\usepackage{float}
\usepackage{url}
\usepackage{csquotes}
\usepackage{fourier} %font
\usepackage[top=2cm, bottom=2cm, left=2.5cm, right=2.5cm]{geometry}
\usepackage{pgfplots}
\usepackage{fancyhdr}
\usepackage{mdframed}
\usepackage{tikzducks}
\usepackage[nameinlink]{cleveref}
\usepackage{epigraph} 

\pgfplotsset{compat=1.18}

\usetikzlibrary{shapes.arrows, shapes.geometric, arrows.meta,angles,quotes,positioning,arrows,fit,quotes,calc}
\tikzset{>=latex} 

\setlength\algomargin{1em} 
\SetFuncSty{sc} 
\SetCommentSty{em} 


\Crefname{figure}{Fig.}{Figs.}
\newcommand\crefrangeconjunction{--}
\Crefname{table}{Tabla}{Tablas}
\Crefname{subsubsection}{Subsubsec.}{Subsubsections}
\Crefname{subsection}{Subsec.}{Subsections}
\Crefname{section}{Sec.}{Sections}
\Crefname{equation}{eq.}{eqs.}
\crefname{thm}{Theorem}{theorems}
\Crefname{thm}{Theorem}{Theorems} 

\definecolor{algoco}{rgb}{0,0.0,1}

\hypersetup{
  colorlinks=true,
  linkcolor=algoco,
  citecolor=blue,
  urlcolor=blue,
}

\lstset{
extendedchars=true
inputencoding=utf8/latin1,
basicstyle=\footnotesize\sffamily\color{black},
commentstyle=\slshape \color{gray},
numbers=left,
numbersep=10pt,
numberstyle=\tiny\color{red!80!black},
keywordstyle=\color{red!80!magenta},
showspaces=false,
showstringspaces=false,
stringstyle=\color{cyan!80!black},
tabsize=2,
literate={á}{{\'a}}1 {é}{{\'e}}1 {í}{{\'i}}1 {ó}{{\'o}}1 {ú}{{\'u}}1,
frame = single, 
numbers = none,
float, floatplacement = ht, captionpos = b,
xleftmargin = 2em, xrightmargin = 2em, 
}

\newcommand{\ub}[1]{\underbrace{#1}}
\newcommand\tcm{\textbf}
\newcommand\tca{\textcolor{algoco}}

\setlength\epigraphwidth{.7\textwidth} 

\newcommand{\tnum}{1} % reemplace 1 por el número de la tarea
\newcommand{\sem}{2025-2} % reemplace 2024-2 por el semestre correspondiente
\newcommand{\campus}{San Joaquin} % reemplace Casa Central por el campus correspondiente
\newcommand{\rolusm}{202173648-4} % reemplace 2025073100-1 por su rol
\newcommand{\namestudent}{Agustin Concha Gordillo} % reemplace Al Goritmo Pérez por su nombre
\newcommand{\deadline}{08 de septiembre de 2025 } % reemplace 26 de abril de 2025, medio día por la fecha de entrega


\headheight=14pt
\linespread{1.3}
\author{\namestudent}
\pagestyle{fancy}
\fancyhf{}%
\fancyfoot[R]{ \namestudent \\ \rolusm}
\fancyfoot[L]{Campus \campus} 
\fancyfoot[C]{\thepage}
\rhead{\sem}
\lhead{INF-221}
\renewcommand{\headrulewidth}{0.4pt}
\renewcommand{\footrulewidth}{0.4pt}
\newbool{programs}
\boolfalse{programs}
\chead{REPORTE TAREA \tnum~}



\title{
  \huge
  \textbf{REPORTE TAREA \tnum~ \\ ALGORITMOS Y COMPLEJIDAD} \\[1ex]
  \emph{\textquote{Más allá de la notación asintótica: Análisis experimental de algoritmos de ordenamiento y multiplicación de matrices.}} 
  }

  
\date{
  \small
  \today\\
  \currenttime
}




\begin{document}
\maketitle
\thispagestyle{fancy} 
\vspace{-1.0\baselineskip}




\begin{abstract}
  \textit{ 
    
/section{Resumen}
Este informe consiste en la documentaci\'on del desarrollo de una serie de algoritmos de ordenamiento, implementados en C++. El objetivo principal es comparar su rendimiento y eficiencia en diferentes escenarios y conjuntos de datos. Se han implementado los siguientes algoritmos: Quicksort,
Mergesort, Insertion Sort, Pandasort y un algoritmo de ordenamiento nativo de C++, adem\'as de dos algoritmos de multiplicacion de matrices el "ingenuo" y strassen. 
A lo largo del informe, se describen brevemente los principios de funcionamiento de cada algoritmo, así como su complejidad temporal y espacial teórica. Posteriormente, se presentan experimentos prácticos para evaluar el desempeño de los algoritmos bajo diferentes condiciones, como el tamaño y la naturaleza de los datos de entrada.
Finalmente, se discuten los resultados obtenidos, destacando las ventajas y desventajas de cada enfoque, y se proponen posibles mejoras o aplicaciones futuras.
  }
     
\end{abstract}

\setcounter{tocdepth}{1}
\tableofcontents


\newpage
\section{Introducción}
\section{Introducción}
El análisis de algoritmos es una disciplina fundamental en la informática, ya que permite evaluar la eficiencia y el comportamiento de los métodos utilizados para resolver problemas computacionales.
A través del estudio de algoritmos, es posible comparar diferentes enfoques para una misma tarea, identificar posibles mejoras y seleccionar la solución más adecuada según los recursos disponibles y las restricciones del problema.
En este informe se abordará el análisis de diversos algoritmos, considerando tanto su diseño teórico como su implementación práctica. Se explorarán aspectos como la complejidad temporal y espacial, así como el impacto de las decisiones de implementación en el rendimiento real de los algoritmos.
Además, se presentarán experimentos y resultados que permitirán contrastar el comportamiento esperado con el observado en la práctica.
El objetivo principal de este trabajo es proporcionar una visión integral sobre el proceso de análisis y comparación de algoritmos, destacando la importancia de una evaluación rigurosa para el desarrollo de soluciones eficientes y robustas en el ámbito de la programación y la ingeniería de software.

\newpage
\section{Implementaciones}
/section{Implementaciones}
 
 El código fuente de las implementaciones desarrolladas para este informe está disponible en el siguiente repositorio de GitHub:

 \url{https://github.com/INF221-20252/tarea-1-CatriX92.git}

\newpage
\section{Experimentos}
\section*{Descripción de los Experimentos}

Para los experimentos realizados en este trabajo, se utilizaron dos conjuntos de datos principales: \textbf{Dataset A} y \textbf{Dataset B}. El \textbf{Dataset A} consiste en instancias pequeñas y medianas generadas artificialmente, mientras que el \textbf{Dataset B} contiene instancias tomadas de problemas reales, lo que permite evaluar el desempeño de los algoritmos en diferentes contextos.

La infraestructura utilizada fue un computador personal con procesador Intel Core i5-13400F, 32 GB de memoria RAM y disco SSD, ejecutando Ubuntu 22.04 LTS. El código fue implementado en C++ y compilado con g++ versión 13.3.0. Para la medición de tiempos de ejecución se utilizó la función \texttt{std::chrono} de la biblioteca estándar.

Se realizaron varias ejecuciones para cada instancia, registrando el tiempo de ejecución y el consumo de memoria. Los parámetros de los algoritmos fueron seleccionados tras pruebas preliminares, buscando un equilibrio entre calidad de resultados y tiempos razonables de cómputo. En la siguiente sección se presentan los resultados obtenidos, acompañados de tablas y gráficos que facilitan la comparación entre los métodos evaluados.

Cabe destacar que, aunque se intentó mantener condiciones controladas, pueden existir pequeñas variaciones en los resultados debido a procesos en segundo plano del sistema operativo y a la naturaleza de los algoritmos aleatorizados.

\subsection{Dataset (casos de prueba)}
/subsection{Descripción del Dataset}
 
 Para los experimentos realizados en este trabajo, se utilizaron los siguientes casos de prueba proporcionados en el enunciado. Cada archivo de entrada tiene el formato \texttt{\{n\}\_\{t\}\_\{d\}\_\{m\}.txt}, donde:

\begin{itemize}
    \item $n$ es la cantidad de elementos del arreglo, con valores en $N = \{10**1, 10**3, 10**5, 10**7\}$.
    \item $t$ indica el tipo de arreglo: ascendente, descendente o aleatorio.
    \item $d$ representa el dominio de los elementos: $D1 = \{0,1,2,3,4,5,6,7,8,9\}$ o $D7 = \{0,1,2,\ldots,107\}$.
    \item $m$ corresponde a la muestra aleatoria o caso de prueba, con valores en $M = \{a, b, c\}$.
\end{itemize}

La salida esperada es un archivo \texttt{\{n\}\_\{t\}\_\{d\}\_\{m\}\_out.txt} con la matriz resultante.

Estos casos de prueba cubren diferentes tamaños de entrada, tipos de ordenamiento y dominios de valores, lo que permite evaluar el comportamiento de los algoritmos bajo distintas condiciones. Sin embargo, la cantidad de tamaños y dominios es limitada, lo que podría restringir la generalización de los resultados.
Además, al no permitir la incorporación de nuevos casos, el análisis experimental se limita a la diversidad propuesta en el enunciado, lo cual es suficiente para comparar algoritmos en un contexto controlado, pero podría no reflejar todos los escenarios posibles en aplicaciones reales.

\subsection{Resultados}
\subsection*{Datos de Ejecución de los Algoritmos}

\begin{table}[H]
    \centering
    \caption{Resultados de ejecución de los algoritmos}
    \label{tab:resultados_ejecucion}
    \begin{tabular}{|c|c|c|}
        \hline
            \textbf{Algoritmo} & \textbf{Tiempo (s)} & \textbf{Espacio (MB)} \\
            \hline
            Quicksort & [-] & 38.147 \\
            \hline
            Pandasort & 5655.73 & 38.147 \\
            \hline
            Insertionsort & [-] & 38.147 \\
            \hline
            Mergesort & 85.6627 & 38.147 \\
            \hline
            Sort & 65.0867 & 38.147 \\
            \hline
            Naive & [-] & [-] \\
            \hline
            Strassen & [-] & [-] \\
            \hline
    \end{tabular}
\end{table}

\subsection*{Ejecuciones Incompletas o No Finalizadas}

En algunos casos, ciertos algoritmos no lograron completar su ejecución debido a limitaciones de tiempo, memoria, o errores inesperados durante el proceso experimental. A continuación, se detallan los algoritmos afectados, las condiciones bajo las cuales no finalizaron y una breve explicación de las posibles causas:

\begin{itemize}
    \item \textbf{[Quicksort]}: No finalizó debido a una mala implementación que causaba un desbordamiento de pila en entradas muy grandes.
    \item \textbf{[Insertionsort]}: No finalizó debido a un tiempo de ejecución excesivo para entradas mayores a 10,000 elementos.
    \item \textbf{[Strassen]}: No finalizó debido a errores de lectura de los archivos de entrada.
    \item \textbf{[Naive]}: No finalizó debido a errores de lectura de los archivos de entrada.
\end{itemize}
\subsection*{Discusión Teórica y Comparación Práctica}

Desde un punto de vista teórico, los algoritmos de ordenamiento como Quicksort y Mergesort suelen ser preferidos para grandes volúmenes de datos debido a su complejidad temporal promedio de \(O(n \log n)\).
Quicksort es conocido por su eficiencia en la práctica, aunque su peor caso es \(O(n^2)\), mientras que Mergesort garantiza \(O(n \log n)\) en todos los casos, a costa de un mayor uso de memoria. Insertionsort, por otro lado, es adecuado solo para conjuntos de datos pequeños debido a su complejidad \(O(n^2)\).
En la práctica, según los resultados obtenidos, Mergesort y Sort (la función de ordenamiento nativa) completaron la tarea en tiempos razonables, validando la teoría. Pandasort, aunque funcional, fue considerablemente más lento, probablemente debido a la sobrecarga de la biblioteca.
Quicksort e Insertionsort no finalizaron, lo que evidencia que una mala implementación o la elección de un algoritmo inadecuado puede afectar drásticamente el desempeño real, independientemente de la teoría. Finalmente, los algoritmos Naive y Strassen presentaron problemas de entrada, por lo que no fue posible evaluarlos.
En conclusión, aunque la teoría sugiere ciertos algoritmos como óptimos, la implementación y el contexto práctico pueden modificar significativamente los resultados esperados.
\noindent
\textbf{Nota:} 

\newpage
\section{Conclusiones}
\section{Conclusiones}
    \begin{itemize}
        \item Los resultados obtenidos demuestran que la metodología propuesta es efectiva para abordar el problema planteado, permitiendo una mejora significativa respecto a los enfoques tradicionales.
        \item El análisis realizado evidencia que las soluciones implementadas no solo cumplen con los objetivos iniciales, sino que también aportan una perspectiva novedosa sobre el tema, facilitando futuras investigaciones o aplicaciones prácticas.
        \item Se observa que las limitaciones identificadas durante el desarrollo del trabajo abren nuevas oportunidades para optimizar el enfoque, sugiriendo líneas de investigación adicionales que podrían fortalecer los resultados alcanzados.
        \item En síntesis, el trabajo contribuye al entendimiento del problema al ofrecer evidencia concreta sobre la viabilidad y el impacto de la propuesta, respaldando la importancia de continuar explorando este tipo de soluciones en contextos similares.
        \item Como aspecto a mejorar, se reconoce que la gestión del tiempo durante el desarrollo del trabajo no fue óptima, lo que limitó la posibilidad de profundizar en ciertos análisis o explorar alternativas adicionales. Esta autocrítica permite identificar la importancia de una mejor planificación en futuros proyectos, sin desmerecer los logros alcanzados en el presente informe.
    \end{itemize}

\newpage

\newpage
\appendix


\section{Apéndice 1}
\input{sections/appendix1}
\printbibliography

\end{document}


