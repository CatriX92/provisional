\section{Introducción}
El análisis de algoritmos es una disciplina fundamental en la informática, ya que permite evaluar la eficiencia y el comportamiento de los métodos utilizados para resolver problemas computacionales.
A través del estudio de algoritmos, es posible comparar diferentes enfoques para una misma tarea, identificar posibles mejoras y seleccionar la solución más adecuada según los recursos disponibles y las restricciones del problema.
En este informe se abordará el análisis de diversos algoritmos, considerando tanto su diseño teórico como su implementación práctica. Se explorarán aspectos como la complejidad temporal y espacial, así como el impacto de las decisiones de implementación en el rendimiento real de los algoritmos.
Además, se presentarán experimentos y resultados que permitirán contrastar el comportamiento esperado con el observado en la práctica.
El objetivo principal de este trabajo es proporcionar una visión integral sobre el proceso de análisis y comparación de algoritmos, destacando la importancia de una evaluación rigurosa para el desarrollo de soluciones eficientes y robustas en el ámbito de la programación y la ingeniería de software.