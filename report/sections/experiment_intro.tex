\section*{Descripción de los Experimentos}

Para los experimentos realizados en este trabajo, se utilizaron dos conjuntos de datos principales: \textbf{Dataset A} y \textbf{Dataset B}. El \textbf{Dataset A} consiste en instancias pequeñas y medianas generadas artificialmente, mientras que el \textbf{Dataset B} contiene instancias tomadas de problemas reales, lo que permite evaluar el desempeño de los algoritmos en diferentes contextos.

La infraestructura utilizada fue un computador personal con procesador Intel Core i5-13400F, 32 GB de memoria RAM y disco SSD, ejecutando Ubuntu 22.04 LTS. El código fue implementado en C++ y compilado con g++ versión 13.3.0. Para la medición de tiempos de ejecución se utilizó la función \texttt{std::chrono} de la biblioteca estándar.

Se realizaron varias ejecuciones para cada instancia, registrando el tiempo de ejecución y el consumo de memoria. Los parámetros de los algoritmos fueron seleccionados tras pruebas preliminares, buscando un equilibrio entre calidad de resultados y tiempos razonables de cómputo. En la siguiente sección se presentan los resultados obtenidos, acompañados de tablas y gráficos que facilitan la comparación entre los métodos evaluados.

Cabe destacar que, aunque se intentó mantener condiciones controladas, pueden existir pequeñas variaciones en los resultados debido a procesos en segundo plano del sistema operativo y a la naturaleza de los algoritmos aleatorizados.
