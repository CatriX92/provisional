\subsection*{Datos de Ejecución de los Algoritmos}

\begin{table}[H]
    \centering
    \caption{Resultados de ejecución de los algoritmos}
    \label{tab:resultados_ejecucion}
    \begin{tabular}{|c|c|c|}
        \hline
            \textbf{Algoritmo} & \textbf{Tiempo (s)} & \textbf{Espacio (MB)} \\
            \hline
            Quicksort & [-] & 38.147 \\
            \hline
            Pandasort & 5655.73 & 38.147 \\
            \hline
            Insertionsort & [-] & 38.147 \\
            \hline
            Mergesort & 85.6627 & 38.147 \\
            \hline
            Sort & 65.0867 & 38.147 \\
            \hline
            Naive & [-] & [-] \\
            \hline
            Strassen & [-] & [-] \\
            \hline
    \end{tabular}
\end{table}

\subsection*{Ejecuciones Incompletas o No Finalizadas}

En algunos casos, ciertos algoritmos no lograron completar su ejecución debido a limitaciones de tiempo, memoria, o errores inesperados durante el proceso experimental. A continuación, se detallan los algoritmos afectados, las condiciones bajo las cuales no finalizaron y una breve explicación de las posibles causas:

\begin{itemize}
    \item \textbf{[Quicksort]}: No finalizó debido a una mala implementación que causaba un desbordamiento de pila en entradas muy grandes.
    \item \textbf{[Insertionsort]}: No finalizó debido a un tiempo de ejecución excesivo para entradas mayores a 10,000 elementos.
    \item \textbf{[Strassen]}: No finalizó debido a errores de lectura de los archivos de entrada.
    \item \textbf{[Naive]}: No finalizó debido a errores de lectura de los archivos de entrada.
\end{itemize}
\subsection*{Discusión Teórica y Comparación Práctica}

Desde un punto de vista teórico, los algoritmos de ordenamiento como Quicksort y Mergesort suelen ser preferidos para grandes volúmenes de datos debido a su complejidad temporal promedio de \(O(n \log n)\).
Quicksort es conocido por su eficiencia en la práctica, aunque su peor caso es \(O(n^2)\), mientras que Mergesort garantiza \(O(n \log n)\) en todos los casos, a costa de un mayor uso de memoria. Insertionsort, por otro lado, es adecuado solo para conjuntos de datos pequeños debido a su complejidad \(O(n^2)\).
En la práctica, según los resultados obtenidos, Mergesort y Sort (la función de ordenamiento nativa) completaron la tarea en tiempos razonables, validando la teoría. Pandasort, aunque funcional, fue considerablemente más lento, probablemente debido a la sobrecarga de la biblioteca.
Quicksort e Insertionsort no finalizaron, lo que evidencia que una mala implementación o la elección de un algoritmo inadecuado puede afectar drásticamente el desempeño real, independientemente de la teoría. Finalmente, los algoritmos Naive y Strassen presentaron problemas de entrada, por lo que no fue posible evaluarlos.
En conclusión, aunque la teoría sugiere ciertos algoritmos como óptimos, la implementación y el contexto práctico pueden modificar significativamente los resultados esperados.
\noindent
\textbf{Nota:} 